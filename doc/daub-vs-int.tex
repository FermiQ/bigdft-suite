\documentclass[12pt]{article}

%\usepackage{pslatex}
%------------------------------------------------------------------------
\usepackage{amsmath,amsfonts}

%------------------------------------------------------------------------
%We start the document
\begin{document}

\section{Relation between Daubechies and Interpolating scaling functions in BigDFT code}
We know that the definition of an interpolating scaling function (ISF) of order $2m-1$ (with $2m-1$ vanishing moments\footnote{Note that the definition is in this case shifted by one wrt the definition we use in {\tt BigDFT} since for the ISF of the Poisson Solver the moments $M_\ell=\delta_{l,m}$, $\ell=0,\cdots m$.}) can be obtained from the autocorrelation of the Daubechies scaling function (DSF) of order $m$:
\begin{equation}
 \varphi(x) =\int \phi(t) \phi(t-x) {\rm d}t\;.
\end{equation}
In the {\tt BigDFT} code we use a formally different relation between the ISF and the DSF.
Indeed, a given function $f(x)$ can be expressend both in ISF or in DSF:
\begin{align}
 f(x)&= \sum_i f^D_i \phi_i(x) \notag \\
 &= \sum_i f^I_i \varphi_i(x)\;,
\end{align}
with the magic filter relation $f^D_i=\sum_j \omega_{i,j} f^I_j$. Hence
\begin{equation}
 f(x)=\sum_{i,j} \omega_{i,j} f^I_j \phi_i(x)\;.
\end{equation}
For the two relations to be compatible, one must have the filters $\omega$ to be optimal quadrature coefficients for the autocorrelation integral:
\begin{equation}
\sum_{i} \omega_{i,j} \phi_i(x) = \int \phi(t) \phi(t-x+j) {\rm d}t\;.
\end{equation}
If this is true, we can have an exact relationship between DSF of order 8 and ISF of order 16 (in {\tt BigDFT} notation).

The refinement relation can help us in finding the coefficients $\omega_i$.
Indeed,
\begin{equation}
 \omega_{i,j} = \int \phi(t) \phi(t-x+j) \phi(x-i) {\rm d}t {\rm d}x = \omega_{i-j}\;.
\end{equation}
By using the refinement relation it is easy to show that
\begin{align}
 \omega_{i}& = \int {\rm d}t {\rm d}x \; \phi(t) \phi(t-x+i) \phi(x) \notag \\
&= \frac{1}{2} \sum_{j,k} h_j h_k \int {\rm d}t {\rm d}x\; \phi(2t-j) \phi(t-x+i) \phi(2x-k) \notag \\
&= \frac{1}{8} \sum_{j,k} h_j h_k \int {\rm d}t {\rm d}x \;\phi(t-j) \phi(\frac{t-x}{2}+i) \phi(x-k) \notag \\
&= \frac{1}{8 \sqrt 2} \sum_{j,k,\ell} h_j h_k h_\ell \int {\rm d}t {\rm d}x\; \phi(t-j) \phi(t-x+2 i-\ell) \phi(x-k)  \\
 &= \frac{1}{8 \sqrt 2} \sum_{j,k,\ell} h_j h_k h_\ell\; \omega_{2i +j -k -\ell} \notag \;.
\end{align}
This relation may help us to find the value of the $\omega_i$ by following the same reasoning has for the kinetic energy filters.
\end{document}


