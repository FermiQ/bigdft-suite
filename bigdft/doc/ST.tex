\documentclass[a4paper,10pt]{article}
\usepackage[utf8x]{inputenc}

%opening

\title { Finite temperature correction} 
\author {ali.sadeghi@unibas.ch}
\begin{document}
\maketitle


When the electronic temperature is finite the free energy, but not the energy, of the ground state is stationary.
So a correction term  
\[
 \Delta u = -\sum_i^{N_e} \int \epsilon_i df_i
   = - \frac{k_B T_e}{2 \sqrt{\pi}} \sum_i^{N_e} \exp \left(-[erf^{-1}(1-2f_i)]^2\right) - N_e \epsilon_F+cte.
\]
has to be added to the energy. The forces, however can be shown to remain unchanged.
Effectively, we can use 
\[
\Delta u = - \frac{k_B T_e}{2 \sqrt{\pi}} \sum_i^{N_e} \exp \left(-[erf^{-1}(1-2f_i)]^2\right).
\]
To avoid eventual numerical difficulties one might face using the  $err^{-1}$ function,  one can use  
\[(1-2f_i)=erf(\frac{\epsilon_i-\epsilon_F}{k_B T_e}) \] 
to express the correction term versus the energy eigen values
\[ {\Delta u = - \frac{k_B T_e}{2 \sqrt{\pi}} \sum_i^{N_e} \exp [-(\frac{\epsilon_i-\epsilon_F}{k_B T_e})^2]}.
\]

\end{document}
